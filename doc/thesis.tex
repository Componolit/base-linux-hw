\documentclass[
a4paper,
12pt,
notitlepage,
parskip=half,
DIV=11,
]{scrbook}

\usepackage[ngerman]{babel}
\usepackage[utf8]{inputenx}
\usepackage[T1]{fontenc}
\usepackage{hyperref}

\begin{document}
	
	\chapter{Background}
	\chapter{Present situation} %%related work
	\chapter{Required steps} %%what needs to be done
	\chapter{Planned design}
		\section{Integration into Genode}
		The integration into Genode is done by creating a base directory.
		This is used to build a core binary which then is executed by the kernel.
		Since there is already a core that is able to use the Linux kernel it would be redundant to rewrite these parts.
		Reusing the already available code can be done in three different ways.
		\subsection{Copy base-linux}
		The simplest way to build upon base-linux is creating a copy it in a new base directory.
		The advantage of this approach lies in the fact that there is, beside including the new directory, no need to modify upstream code.
		On the other hand this leads to code duplication and any updates of base-linux probably need to be ported to its modified copy.
		\subsection{Include base-linux}
		Creating a new base directory and including the needed contents from base-linux via Makefile fragments.
		While this would include the advantages of the copy based approach there is less to no code duplication.
		The disadvantage here is the complexity due to the build system that does not provide a standard way to include parts of other base directories.
		\subsection{Extend base-linux}
		To prevent both code duplication and complex includes the already existing base directory could be used and extended.
		While this approach seems to be easier it has two caveats.
		Since it requires modifications of the original base-linux core the current functionality needs to be maintained.
		This can be done either by creating a hybrid core that is able to run like the current one and on a bare Linux kernel or a switch needs to be added that chooses the target on build time.
		Additionally this code needs to be brought upstream as it cannot be plugged into the source tree like a separate base directory.
	\chapter{Implemented design}
	\chapter{Evaluation}
	\chapter{Conclusion}
\end{document}